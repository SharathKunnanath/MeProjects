\documentclass{article}
\usepackage{amsmath}

\title{The Fibonacci Sequence}
\author{S.Kunnanath}
\date{\today}

\begin{document}

\maketitle

\section{Introduction}
The Fibonacci sequence is a series of numbers in which each number (Fibonacci number) is the sum of the two preceding ones. The sequence typically starts with 0 and 1. 

\section{Definition}
Mathematically, the Fibonacci sequence is defined by the recurrence relation:
\[
F(n) = F(n-1) + F(n-2)
\]
with initial conditions:
\[
F(0) = 0, \quad F(1) = 1
\]

\section{First Few Terms}
The first few terms in the Fibonacci sequence are:
\[
0, 1, 1, 2, 3, 5, 8, 13, 21, 34, \ldots
\]

\section{Problem Statement}
Write a function that utilizes dynamic programming to compute the \(n\)-th Fibonacci number efficiently. 
Write a function for both the recursive and iterative dynamic programming solution 

\end{document}